\documentclass{homework}
\usepackage{bussproofs}
\EnableBpAbbreviations
\usepackage{ctex,hyperref,float,algorithm,algorithmic}
\hypersetup{hidelinks,
	colorlinks=true,
	allcolors=blue,
	pdfstartview=Fit,
	breaklinks=true
}

\name{熊泽恩} % Replace (Student Name) with your name.
\id{2022011223}
\term{2024 Spring}
\course{Introduction to Theoretical Computer Science}
\hwnum{8}

%\hwname{(Name)}          % Uncomment and replace (Name) with the type of the
                          % homework (e.g, Assignment, Problem Set, etc.) if you
                          % don't want the document to be labeled as "Homework."
%\problemname{(Name)}     % Uncomment and replace (Name) with the desired label
                          % for problems created with the problem environment.
%\solutionname{(Name)}    % Uncomment and replace (Name) with the desired label
                          % for solutions created with the solution environment.

% Load any other packages you need here.

\begin{document}

\begin{problem}
  Two aliens have arrived on Earth, each claiming to possess a machine that can
  solve the $\SAT$ problem in polynomial time.
  However, only one of these machines is genuine, while the other is a
  counterfeit.
  Your task is to design a protocol for solving the $\SAT$ problem in polynomial
  time by asking questions to their machines.
\end{problem}

\begin{solution}
  
  Let the two aliens be $A$ and $B$,
  and their machines be $M_{A}$ and $M_{B}$ respectively.
  Suppose $M_{A}$ corresponds to a algorithm $A$
  which can decide whether $\varphi(x_1, x_2, \cdots, x_n)$
  is satisfiable or not in polynomial time.
  
  Consider algorithm $S$ as follows:
  \begin{enumerate}
    \item For $i = 1, 2, \cdots, n$ do:
    \begin{enumerate}
      \item Use $M_{A}$ to check whether
        $\varphi(a_1, \cdots, a_{i-1}, 0, x_{i+1}, \cdots, x_n)$
        is satisfiable or not.
      \item If so, set $a_i = 0$. Otherwise, set $a_i = 1$.
    \end{enumerate}
    \item Check if $a = (a_1, a_2, \cdots, a_n)$ is a ``satisfying'' assignment for $\varphi$.
      Output $a$ if it is.
  \end{enumerate}

  Since one of the machines is a counterfeit,
  we couldn't guarantee that the output is a truly satisfying assignment.
  However, $\SAT \in \NP$, so we can verify the output in polynomial time.
  If the output is a truly satisfying assignment, then $M_A$ is the genuine machine.
  Otherwise, $M_B$ is the genuine machine.

  Now that we have decided which machine is genuine,
  we can use it to solve the $\SAT$ problem in polynomial time.

\end{solution}

% \begin{problem}
%   Prove that if $A \in \P$, then $\P^A = \P$.
% \end{problem}

% \begin{solution}

% \end{solution}

% \begin{problem}
%   \begin{parts}
%     \part\label{a} Let $C \subseteq {\{0,1\}}^{*}$ be a language.
%     Define another unary language
%     \begin{equation*}
%       L(C) = \{ 1^{n} \mid \text{all strings } x \text{ of length } n
%         \text{ is in } C \}.
%     \end{equation*}
%     Prove that $L(C) \in \coNP^{C}$ for all $C$.
%     \part\label{b}
%     A DNF formula is the logical OR of terms where each term is the logical AND
%     of literals.
%     The width of a DNF formula is the maximum number of literals in a term and
%     the size of of it is the number of terms.
%     Prove that $x_{1} \land x_{2} \land \cdots \land x_{N}$ cannot be computed
%     by a DNF formula $\varphi(x_{1}, x_{2}, \ldots, x_{N})$ of width less then
%     $N$.
%     \part\label{c} Use \cref{b} to construct an oracle $C$ such that
%     $L(C) \not\in \NP^{C}$ thereby showing that $\NP^{C} \ne \coNP^{C}$.
%   \end{parts}
% \end{problem}

% \begin{solution}

% \end{solution}

\end{document}
