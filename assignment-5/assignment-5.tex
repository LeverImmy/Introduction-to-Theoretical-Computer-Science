\documentclass{homework}
\usepackage{bussproofs}
\EnableBpAbbreviations
\usepackage{ctex}

\name{熊泽恩} % Replace (Student Name) with your name.
\id{2022011223}
\term{2024 Spring}
\course{Introduction to Theoretical Computer Science}
\hwnum{5}

%\hwname{(Name)}          % Uncomment and replace (Name) with the type of the
                          % homework (e.g, Assignment, Problem Set, etc.) if you
                          % don't want the document to be labeled as "Homework."
%\problemname{(Name)}     % Uncomment and replace (Name) with the desired label
                          % for problems created with the problem environment.
%\solutionname{(Name)}    % Uncomment and replace (Name) with the desired label
                          % for solutions created with the solution environment.

% Load any other packages you need here.

\begin{document}

\begin{problem}
  Prove the following corollary of the time hierarchy theorem: for all constants
  $c_1 > c_0 \ge 1$, $\TIME(n^{c_0}) \subsetneq \TIME(n^{c_1})$.
\end{problem}

\begin{solution}

  First of all, we have the following lemma:
  \begin{lemma}
    For all constants $c_1 > c_0 \ge 1$, $n^{c_0} = o(n^{c_1}/\log n)$.
  \end{lemma}
  \begin{proof}
    Since $c_1 > c_0 \ge 1$, we have
    \begin{equation*}
      \lim_{n \to \infty} \frac{n^{c_0}}{n^{c_1}/\log n}
      = \lim_{n \to \infty} \frac{\log n}{n^{c_1 - c_0}}
      = 0.
    \end{equation*}
    Therefore, $n^{c_0} = o(n^{c_1}/\log n)$.
  \end{proof}
  
  From the hierarchy theorem,
  we have $\TIME(o(n^{c_1}/\log(n^{c_1}))) \subsetneq \TIME(n^{c_1})$.
  That is to say, $\TIME(o(n^{c_1}/\log n)) \subsetneq \TIME(n^{c_1})$.
  $n^{c_0} = o(n^{c_1}/\log n)$ from the lemma above,
  therefore, we have $\TIME(n^{c_0}) \subsetneq \TIME(n^{c_1})$.
  
\end{solution}

\begin{problem}
  \begin{parts}
    \part\label{2a}
    Show that $\P$ is closed under union, concatenation, and complement.
    That is, $A \cup B, A \circ B, A^{\text{c}} \in \P$ if $A, B \in \P$.
    Note that the concatenation $A \circ B$ of two languages $A$ and $B$ is
    defined as
    \begin{equation*}
      A \circ B = \{ x y \mid x \in A, y \in B\}.
    \end{equation*}
    \part\label{2b}
    Show that $\P$ is closed under the star operation.
    That is, $A^{*} \in \P$ if $A \in \P$ where
    $A^{*} = \{x_{1} x_{2} \cdots x_{k} \mid k \ge 0, x_{j} \in A \text{ for
    } j=1, 2, \ldots, k\}$.
    (Hint: You may need to use dynamic programming to maintain a table whose
    $i,j$-th entry indicates whether $x_{i} \cdots x_{j} \in A^{*}$)
  \end{parts}
\end{problem}

\begin{solution}

  \ref{2a} The proof is as follows:
  \begin{enumerate}
    \item $A \cup B$ is in $\P$, because we can decide whether $x \in A \cup B$
      by first checking whether $x \in A$ and then checking whether $x \in B$.
      If either of them is true, then $x \in A \cup B$.
      Since we can solve both $A$ and $B$ in polynomial time using deterministic
      Turing machines, so first solving $A$ then $B$ consequently will also be
      in polynomial time. Therefore, $A \cup B \in \P$.
    \item $A \circ B$ is in $\P$, because we can decide whether $x \in A \circ B$ by
      checking whether $x = yz$ where $y \in A$ and $z \in B$.
      We can iterate $i$ from $0$ to $|x|$ and check whether $x[0 \ldots i-1] \in A$
      and $x[i \ldots |x| - 1] \in B$.
      Since we can solve both $A$ and $B$ in polynomial time using deterministic Turing machines,
      $A \circ B \in \P$.
    \item $A^{\text{c}}$ is in $\P$, because we can decide whether $x \in A^{\text{c}}$ by
      by negating the result of $x \in A$. Since we can solve $A$ in polynomial time
      using deterministic Turing machines, so solving $A^{\text{c}}$ will also be in
      polynomial time. Therefore, $A^{\text{c}} \in \P$.
  \end{enumerate}

  \ref{2b}
  We can use dynamic programming to maintain a table whose $i,j$-th entry indicates
  whether $x_{i} \cdots x_{j} \in A^{*}$.
  The algorithm is as follows:
  \begin{enumerate}
    \item Initialize the table $T$ with $T_{i, i} = \begin{cases}
      \text{true}, &\text{if } x_i \in A, \\
      \text{false}, &\text{if } x_i \notin A \end{cases}$ for all $i$.
    \item Iterate $l$ from $1$ to $|x|$.
    \item Iterate $i$ from $1$ to $|x| - l$.
    \item Set $j = i + l$.
    \item Iterate $k$ from $i$ to $j - 1$.
    \item Set $T_{i, j} = T_{i, j} \vee (T_{i, k} \wedge T_{k+1, j})$.
    \item Check if $T_{1, |x|}$.
          If it is true, then $x \in A^{*}$;
          otherwise, $x \notin A^{*}$.
  \end{enumerate}

  The correctness of the algorithm is obvious, since $x \in A^*$
  ensures a possible partition of $x$ into $x_1 \cdots x_k$ where
  $x_i \in A$ for all $i$, and the algorithm will find such a partition;
  and the algorithm will only find such $x$
  because of the way the table is updated,
  which ensures that $T_{i, j}$ is true
  if and only if it can be partitioned into smaller segments
  $x_i \cdots x_j \in A^*$.
  Beside the initialization,
  the time complexity of the algorithm is $O(n^{3})$,
  where $n$ is the length of $x$.
  And the initialization can be done in polynomial time
  because $A$ is in $\P$.
  Therefore, $A^{*} \in \P$.

\end{solution}

\begin{problem}
  Karatsuba algorithm is an efficient algorithm for multiplying two natural
  numbers of $n = 2^{k}$ bits, outperforming the straightforward $O(n^{2})$
  primary-school method.
  The key idea is as follows.
  First, we write the numbers as $a 2^{\ell} + b$ and $c 2^{\ell} + d$ where
  $\ell = n/2$ and $a, b, c, d \in \{0, 1, \ldots, 2^{\ell}-1\}$.
  So the product is
  \begin{equation*}
    ac2^{2\ell} + (ad+bc) 2^{\ell} + bd,
  \end{equation*}
  and this reduces the computation of the product to four multiplications
  ($ac, ad, bc, bd$) of shorter numbers.
  Second, Karatsuba's key idea is that three multiplications suffice for the
  computation as one can first compute $ac, bd$.
  Then the coefficient in front of $2^{\ell}$ can be computed by one extra
  multiplication as $ad+bc = (a+b)(c+d) - ac -bd$.
  Show that Karatsuba's algorithm has time complexity
  $O(n^{\log_{2} 3}) \approx O(n^{1.585})$.
\end{problem}

\begin{solution}

  Let $T(n)$ be the time complexity of Karatsuba's algorithm for multiplying two
  $n$-bit numbers. The problem is divided into three subproblems,
  which involve multiplying three pairs of numbers of
  the size $\left\lceil n/2\right\rceil$.
  In addition, the time complexity of the `conquer' part is linear to $n$,
  since the additions, subtractions,
  and digit shifts can be done in linear time.
  Then we have the following recurrence relation:
  \begin{equation*}
    T(n) = 3T(\left\lceil n/2\right\rceil) + O(n).
  \end{equation*}
  
  From the master theorem, we have $a = 3, b = 2, f(n) = O(n)$, and
  $n^{\log_{b} a} = n^{\log_{2} 3} \approx n^{1.585}$.
  Since $f(n) = O(n) = O(n^{\log_{2} 3-\epsilon})$ for some $\epsilon > 0$,
  we have $T(n) = O(n^{\log_{2} 3}) \approx O(n^{1.585})$.

\end{solution}

\end{document}
