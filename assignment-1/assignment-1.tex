\documentclass{homework}
\usepackage{bussproofs}
\EnableBpAbbreviations
\usepackage{ctex}

\name{熊泽恩} % Replace (Student Name) with your name.
\id{2022011223}
\term{2024 Spring}
\course{Introduction to Theoretical Computer Science}
\hwnum{1}

%\hwname{(Name)}          % Uncomment and replace (Name) with the type of the
                          % homework (e.g, Assignment, Problem Set, etc.) if you
                          % don't want the document to be labeled as "Homework."
%\problemname{(Name)}     % Uncomment and replace (Name) with the desired label
                          % for problems created with the problem environment.
%\solutionname{(Name)}    % Uncomment and replace (Name) with the desired label
                          % for solutions created with the solution environment.

% Load any other packages you need here.

\begin{document}

\begin{problem}
  Decide whether the following is a $\lambda$ term (or an abbreviation of a
  $\lambda$ term).
  If it is not, explain the reason.
  \begin{parts}
    \part\label{1.a}
    $\lambda x . x x x$
    \part\label{1.b}
    $\lambda \lambda x. x $
    \part\label{1.c}
    $\lambda y. (\lambda x. x)$
    \part\label{1.d}
    $\lambda uv.((xxy)xy)y)$
  \end{parts}
\end{problem}

\begin{solution}

  \ref{1.a} % chktex 2
  Yes.

  \ref{1.b} % chktex 2
  No. The reason is that there are 2 consecutive $\lambda$ s,
  but there is only 1 dot(.).

  \ref{1.c}
  Yes.

  \ref{1.d}
  No. The reason is that the rightmost parenthesis does not match.

\end{solution}

\begin{problem}
  Compute the terms represented by the following substitutions:
  \begin{parts}
    \part\label{2.a}
    $(xyz)[y/z]$.
    \part\label{2.b}
    $(\lambda x.x)[y/z]$.
    \part\label{2.c}
    $(\lambda y.xy)[yy/x]$.
  \end{parts}
\end{problem}

\begin{solution}

  \ref{2.a}
  $(xyz).[y/z] = xzz$.

  \ref{2.b}
  $(\lambda x.x)[y/z] = \lambda x.x$.

  \ref{2.c}
  $(\lambda y.xy)[yy/x] = (\lambda u.xu)[yy/x] = \lambda u.(xu[yy/x]) = \lambda u.yyu$.

\end{solution}

\begin{problem}
  Prove the following equalities in the theory of $\lambda\beta$.
  You need to draw the ``proof tree'' using the rules we defined in the lecture.
  \begin{parts}
    \part\label{3.a}
    $\lambda uv.v = \lambda u.(\lambda x.x) (\lambda u.u)$.
    \part\label{3.b}
    $\lambda uv.(\lambda xy.y) uv = \lambda ab.b$.
  \end{parts}
\end{problem}

\begin{solution}
  \ref{3.a}
  \begin{prooftree}
    \AXC{}
    \LeftLabel{($\alpha$)}
    \UIC{$\lambda v.v = \lambda u.u$}
    \AXC{}
    \LeftLabel{($\beta$)}
    \UIC{$(\lambda x.x)(\lambda u.u) = \lambda u.u$}
    \LeftLabel{(trans)}
    \BIC{$\lambda v.v = (\lambda x.x)(\lambda u.u)$}
    \LeftLabel{(abst)}
    \UIC{$\lambda uv.v = \lambda u.(\lambda x.x)(\lambda u.u)$}
  \end{prooftree}

  Therefore $\lambda \beta \vdash \lambda uv.v = \lambda u.(\lambda x.x) (\lambda u.u)$.

  \ref{3.b}
  \begin{prooftree}
    \AXC{}
    \LeftLabel{($\beta$)}
    \UIC{$(\lambda xy.y)uv = v$}
    \LeftLabel{(abst)}
    \UIC{$\lambda uv(\lambda xy.y)uv = \lambda uv.v$}
    \LeftLabel{($\alpha$)}
    \AXC{}
    \UIC{$\lambda uv.v = \lambda ab.b$}
    \LeftLabel{(trans)}
    \BIC{$\lambda uv.(\lambda xy.y) uv = \lambda ab.b$}
  \end{prooftree}

  Therefore $\lambda\beta \vdash \lambda uv.(\lambda xy.y) uv = \lambda ab.b$.

\end{solution}

\begin{problem}
  \begin{parts}
    \part\label{4.a}
    Find a $\lambda$ term $s$ such that the equality $stu = ut$ holds in
    $\lambda\beta$ for all terms $t$ and $u$.
    \part\label{4.b}
    Show that there is a $\lambda$ term $s$ such that for all term $t$,
    $\lambda\beta \vdash st = ss$.
  \end{parts}
\end{problem}

\begin{solution}

  \ref{4.a}
  Let $s = \lambda xy.yx$, then

  \begin{align*}
    stu & = (\lambda xy.yx)tu \\
    & = (\lambda x.(\lambda y.yx))tu \\
    & = (\lambda y.yx)[t/x]u \\
    & = (\lambda y.yt)u \\
    & = yt[u/y] \\
    & = ut.
  \end{align*}

  Therefore $stu = ut$ holds in $\lambda\beta$ for all terms $t$ and $u$,
  and thus $s = \lambda xy.yx$ is what we want.

  \ref{4.b}
  If $s = \lambda x.ss$, then for all term $t$,

  \begin{equation*}
    \lambda\beta \vdash st = (\lambda x.ss)t = ss[t/x] = ss.
  \end{equation*}

  So we need to solve the equation $s = \lambda x.ss$,
  and it is a fixed point of the function

  \begin{equation*}
    f = \lambda yx.yy,
  \end{equation*}

  because $fs = (\lambda yx.yy)s = (\lambda x.yy)[s/y] = \lambda x.ss = s$.
  Using the Y combinator, we can construct one possible answer:

  \begin{align*}
    s & = \mathbf{y}f \\
    & = (\lambda u. f(uu))(\lambda u. f(uu)) \\
    & = (\lambda u. (\lambda yx.yy)(uu))(\lambda u. (\lambda yx.yy)(uu)).
  \end{align*}

\end{solution}

\begin{problem}
  Show that there is a term $G$ such that all fixed-point combinators can be
  \emph{characterized} as the fixed points of $G$.
  That is, $s$ is a fixed-point combinator if and only if
  $\lambda\beta \vdash G s = s$.
\end{problem}

\begin{solution}

  Let $G = \lambda yx.x(yx)$.

  Suppose $s$ is a fixed-point combinator,
  then the equation $sf = f(sf)$ holds for all terms $f$.
  If we bind $f$ as $\lambda x$, $s$ satisfies the equation $s = \lambda x.x(sx)$.
  And we can verify that

  \begin{align*}
    Gs & = (\lambda yx.x(yx))s \\
    & = (\lambda x.x(yx))[s/y] \\
    & = \lambda x.x(sx) \\
    & = s.
  \end{align*}

  On the other hand, suppose $G s = s$, then

  \begin{align*}
    s & = G s \\
    & = (\lambda yx.x(yx))s \\
    & = (\lambda x.x(yx))[s/y] \\
    & = \lambda x.x(sx).
  \end{align*}

  Hence for all terms $f$,
  $sf = (\lambda x.x(sx))f = x(sx)[f/x] = f(sf)$.
  That is, $s$ is a fixed-point combinator.

  QED.

\end{solution}

\end{document}
