\documentclass{homework}
\usepackage{ctex,hyperref,float,algorithm,algorithmic,pgfplots}

\name{熊泽恩} % Replace (Student Name) with your name.
\id{2022011223}
\term{2024 Spring}
\course{Introduction to Theoretical Computer Science}
\hwnum{11}

%\hwname{(Name)}          % Uncomment and replace (Name) with the type of the
                          % homework (e.g, Assignment, Problem Set, etc.) if you
                          % don't want the document to be labeled as "Homework."
%\problemname{(Name)}     % Uncomment and replace (Name) with the desired label
                          % for problems created with the problem environment.
%\solutionname{(Name)}    % Uncomment and replace (Name) with the desired label
                          % for solutions created with the solution environment.

% Load any other packages you need here.

\begin{document}

\begin{problem}
  Prove that the function family
  \begin{equation*}
    \mathcal{H} = \bigl\{ h_{a,b} \mid h_{a,b}(x) = a \cdot x + b, a\in {\{0,1\}}^{k},
    b\in \{0,1\} \bigr\}
  \end{equation*}
  is a pairwise independent hash function family for range $R = \{0,1\}$ and
  domain $U = {\{0,1\}}^{k}$.
\end{problem}

\begin{solution}

  All the computations are under modulo 2.

  \begin{lemma}
    Given $x \in \{0, 1\}^k, x \neq 0$, then
    \begin{equation*}
      \Pr_{a \in \{0, 1\}^k}(a \cdot x = 0) = \frac{1}{2}.
    \end{equation*}
  \end{lemma}

  \begin{proof}
    Assume $x_i = 1$ for some $i \in [0, k - 1]$. Then
    \begin{equation*}
      \Pr_{a \in \{0, 1\}^k}(a \cdot x = 0) = \Pr_{a_i \in \{0, 1\}}(a_i = \sum_{j \neq i}a_jx_j) = \frac{1}{2}.
    \end{equation*}
    \hfill\qedsymbol
  \end{proof}

  It is easy to verify that
  \begin{equation*}
    \Pr_{a \in \{0, 1\}^k, b \in \{0, 1\}}(h(x_1) = y_1) = \frac{1}{|R|} = \frac{1}{2},
  \end{equation*}
  so we only need to prove that, for any two distinct elements $x_1, x_2 \in U = \{0, 1\}^k$,
  and two (possibly equal) elements $y_1, y_2 \in R = \{0, 1\}$, we have
  \begin{equation*}
    \Pr_{a \in \{0, 1\}^k, b \in \{0, 1\}}(h_{a, b}(x_1) + b = y_1
    \text{ and } h_{a, b}(x_2) + b = y_2) = \frac{1}{4}.
  \end{equation*}
  This is equivalent to proving that
  \begin{equation*}
    \Pr_{a \in \{0, 1\}^k, b \in \{0, 1\}}(a \cdot x_1 + b = y_1
    \text{ and } a \cdot x_2 + b = y_2) = \frac{1}{4}.
  \end{equation*}

  Separate $a$ and $b$ as follows:
  \begin{align*}
    \quad & \Pr_{a \in \{0, 1\}^k, b \in \{0, 1\}}(h_{a, b}(x_1) = y_1 \text{ and } h_{a, b}(x_2) = y_2) \\
    & = \Pr_{a, b}(a \cdot x_1 + b = y_1 \text{ and } a \cdot x_2 + b = y_2) \\
    & = \Pr_{a, b}(a \cdot (x_1 \oplus x_2) = y_1 \oplus y_2 \text{ and } b = y_1 \oplus a\cdot x_1) \\
    & = \Pr_{a}(a \cdot (x_1 \oplus x_2) = y_1 \oplus y_2) \cdot \Pr_{b}(b = y_1 \oplus a \cdot x_1 \mid a = a_0).
  \end{align*}
  Now that $x_1 \oplus x_2 \neq 0$, so by the lemma, we know that
  \begin{equation*}
    \Pr_{a}(a \cdot (x_1 \oplus x_2) = y_1 \oplus y_2) = \frac{1}{2}.
  \end{equation*}
  Also, fix $a = a_0$, then $\Pr_{b}(b = y_1 \oplus a \cdot x_1 \mid a = a_0) = \frac{1}{2}$.
  Therefore we have
  \begin{equation*}
    \Pr_{a \in \{0, 1\}^k, b \in \{0, 1\}}(a \cdot x_1 + b = y_1
    \text{ and } a \cdot x_2 + b = y_2) = \frac{1}{4},
  \end{equation*}
  and $\mathcal{H}$ is a pairwise independent hash function family.

\end{solution}

\begin{problem}
  \begin{parts}
    \part\label{2.a} Consider a random walk $X_{0}, X_{1}, X_{2}, \ldots$ on a chain
    of $n+1$ vertices $0, 1, \ldots, n$ with the following transition
    probabilities
    \begin{equation*}
      \Pr(X_{t} = k | X_{t-1} = j) =
      \begin{cases}
        \frac{1}{2} & \text{ if } j \in [1,n-1] \text{ and } k = j \pm 1,\\
        1 & \text{ if } j = 0, k = 1 \text{ or } j=n, k=n,\\
        0 & \text{ otherwise}.
      \end{cases}
    \end{equation*}
    Let $T_{i}$ be the expected number of steps the walk takes to arrive at the
    end vertex $n$ starting with $X_{0} = i$.
    Prove that $T_{i} \le n^{2}$ for all $i \in [0,n]$.
    \part\label{2.b} Consider the following randomized algorithm for $\TWOSAT$
    problems of $n$ variables.
    \begin{algorithmic}[1]
      \STATE{Choose an arbitrary initial assignment.}
      \FOR{$t = 1, 2, \ldots, 2n^{2}$}
        \IF{the current assignment is satisfying}
          \STATE{Accept immediately.}
        \ELSE{}
          \STATE{Choose an arbitrary clause not satisfied.}
          \STATE{Sample one of the two literals uniformly at random.}
          \STATE{Flip the value of the variable in the sampled literal.}
        \ENDIF{}
      \ENDFOR{}
      \STATE{Reject if haven't accepted.}
    \end{algorithmic}
    Use Markov inequality to show that the algorithm will find a satisfying
    solution with probability at least $\frac{1}{2}$ given a yes-instance as
    input.
  \end{parts}
\end{problem}

\begin{solution}

  \ref{2.a}
  From the definition of $T_{i}$, we have
  \begin{equation*}
    T_{i} = 1 + \frac{1}{2}T_{i-1} + \frac{1}{2}T_{i+1}, \quad 1 \le i \le n - 1,
  \end{equation*}
  and
  \begin{equation*}
    T_{0} = 1 + T_{1}, \quad T_{n} = 1.
  \end{equation*}
  We can rewrite the above equations as
  \begin{equation*}
    T_{i} - T_{i-1} = T_{i+1} - T_{i} + 2 \quad 1 \le i \le n - 1.
  \end{equation*}
  Summing up the above equations for $i \in [1, n - 1]$, we have
  \begin{equation*}
    T_{n-1} - T_{0} = T_{n} - T_{1} + 2(n-1).
  \end{equation*}
  Since $T_{0} = 1 + T_{1}$ and $T_{n} = 1$, we have
  \begin{equation*}
    T_{n-1} = 2n-1.
  \end{equation*}
  Now we will prove that $T_{i} = n^2 - i^2$ holds for all $i \in [0, n]$ by induction.

  \begin{itemize}
    \item \textbf{Base case:} $T_{n} = n^2$ and $T_{n - 1} = n^2 - (n - 1)^2 = 2n - 1$.
    \item \textbf{Inductive step:} Suppose that $T_{i} = n^2 - i^2$ holds for
          all $i \ge k \in [1, n - 1]$.
    Then we have $T_{i - 1} = 2T_i - T_{i + 1} - 2 = 2(n^2 - i^2) - (n^2 - (i + 1)^2) - 2 = n^2 - (i - 1)^2$.
    \item \textbf{Conclusion:} By induction, we have $T_{i} = n^2 - i^2$ for all $i \in [1, n]$.
          Consider the special case $i = 0$, we have $T_{0} = 1 + T_{1} = 1 + (n^2 - 1) = n^2$.
          Therefore, we have $T_{i} = n^2 - i^2$ for all $i \in [0, n]$.
  \end{itemize}
  Hence, $T_{i} \le n^2$ for all $i \in [0, n]$.

  \ref{2.b}
  Consider the random variable $X_i$. $X_i$ is the number of steps used to
  reach the state that $i$ variables have the correct assignments.
  Suppose we have sampled the literal $l_{i, 1}$.
  \begin{itemize}
    \item If it already has the correct value, then after we flip its value,
          it would be wrong, ending up with the state $X_{i - 1}$.
    \item If it has the wrong value, then after we flip its value,
          it would be correct, ending up with the state $X_{i + 1}$.
  \end{itemize}
  Each of the cases above has a probability of $\frac{1}{2}$.
  Therefore, it is the same procedure as the random walk in part \ref{2.a}.
  
  The target is to have all the $n$ variables assigned with a correct value.
  And the number of steps to reach the state that all the variables
  have the correct assignments (i.e. the state $X_n$), from the state that initially have $i$
  correct assignments (i.e. the state $X_i$), is $Y_i$. Denote $T_i = \E(Y_i)$
  as its expectation.

  Then, by Markov inequality, we have
  \begin{equation*}
    \Pr(Y_i \ge 2n^2) \le \frac{\E(Y_i)}{2n^2} = \frac{T_i}{2n^2} \le \frac{n^2}{2n^2} = \frac{1}{2}.
  \end{equation*}
  Hence, after $2n^2$ steps, the algorithm will find a satisfying solution
  with probability at least $\frac{1}{2}$ given a yes-instance as input.

\end{solution}

\end{document}
